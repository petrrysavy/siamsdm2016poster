\documentclass[25pt, a0paper, landscape, margin=0mm, innermargin=15mm, blockverticalspace=15mm, colspace=15mm, subcolspace=8mm]{tikzposter}

\title{Geometric Methods of Accelerating Triangle-inequality-based $k$-means}
\author{Petr Ry\v{s}av\'{y} and Greg Hamerly}
\institute{Baylor University, Waco, TX}

\usetheme{Desert}
\usecolorstyle[colorPalette=GreenGrayViolet,colorOne=green!30!black,colorTwo=white,colorThree=black]{Denmark}
\colorlet{titlefgcolor}{black}
\colorlet{titlebgcolor}{white}

\usepackage{files/mycommands}

\begin{document}
\maketitle
\begin{columns}
\column{0.25}

\block[titleoffsety=-100cm,bodyoffsety=-100cm]{}{
Petr Ry\v{s}av\'{y} and Greg Hamerly \\
Department of Computer Science\\
Baylor University\\
Waco, TX 76798-7356 \\
\href{mailto:petr_rysavy@alumni.baylor.edu }{petr\_rysavy@alumni.baylor.edu } \\
\href{mailto:greg_hamerly@baylor.edu}{greg\_hamerly@baylor.edu}
}

\block{Abstract}{
\textit{Most implementations of $k$-means use Lloyd's algorithm, which does {\bf many unnecessary distance calculations}. Several accelerated algorithms produce exactly the same answer by using bounds on point-center distances which are efficiently updated with the triangle inequality.  In this work we propose {\bf tighter lower bound updates} and {\bf efficiently skip centroids} that cannot possibly be close to a set of points. In our experiments, these improvements {\bf accelerate fast algorithms up to eight times faster}.}
}


\block{$k$-means Clustering}{
    \begin{itemize}
    \item Inputs: $n$ points $\left\{ \itemization{\vec{x}}{n} \right\}$;
    number of centroids $k$.
    \item Goal: find a set of $k$ centroids $\left\{ \itemization{\vec{c}}{k}
    \right\}$ to minimize the \emph{distortion function}
      \begin{equation*}
        J(\itemization{\vec{c}}{k}) = \sumion \min_{1 \le j \le k} \left\| \vec{x}_i - \vec{c}_j \right\|^2.
        \label{eq:distortion}
      \end{equation*}
    \end{itemize}
  %(function $\vec{c}$ returns the assigned centroid to the given argument)
  %\begin{figure}[ht]
%\centering%
\begin{center}
\begin{tikzpicture}
	\begin{axis}[ 
		%xlabel=Product $|L| \cdot |E|$,
		%axis y label/.style={at={(current axis.left)}},
		%ylabel=Time required for run of the greedy part in $\mathrm{ms}$,
		cycle list name=color list,
		width=0.15\textwidth,
		only marks,
		yticklabels={,,},
		xticklabels={,,},
		%height=7cm,
		%legend pos= south east,
		%legend cell align=left,
		%xmin=0,
		%ymin=0,
		%ymode=log,
		%log ticks with fixed point,
		] 
		\addplot table[scatter,x=x,y=y] {files/cluster1.dat};%
		\addplot table[scatter,x=x,y=y] {files/cluster2.dat};%
		\addplot[color=green!50!black] table[scatter,x=x,y=y] {files/cluster3.dat};%
		\addplot[color=black, mark=x, mark size=7, line width=3pt] table[scatter,x=x,y=y] {files/centroids.dat};
		%\legend{Tourist trails data,Tram lines data};%
	\end{axis}
\end{tikzpicture}
\end{center}
%\caption[An illustration of the $k$-means problem.]{An illustration of the $k$-means problem.}
%\label{plot:clusters}
%\end{figure}
}

\block{Lloyd's Algorithm \cite{lloyd}}{
  Repeat until convergence:
     \begin{enumerate}
       \item {\bf Assign} each point to its closest centroid
       \item {\bf Move} centroids to the cluster means
     \end{enumerate}
}


\column{0.25}

\block{Triangle-Inequality-Based $k$-means}{
   Lloyd's algorithm calculates all point-center distances, even when centers
   move very little. Several algorithms avoid this redundant work by
   maintaining distance bounds which can prove when cluster assignments could
   not have changed.  The triangle inequality allows efficient and correct but
   loose bound updates.  Such algorithms produce \emph{exactly} the same
   results as Lloyd's algorithm, only \emph{faster}.
   
  \begin{theorem}[Triangle inequality] \label{thm:triange}
    For any vectors $\vec{x}$ and $\vec{y}$,
    \begin{equation*}
       \| \vec{x} + \vec{y} \| \leq \|\vec{x}\| + \|\vec{y}\|.
    \end{equation*}
 \end{theorem}
 \begin{center}
 \tiny
  \usetikzlibrary{calc}
\usetikzlibrary{arrows}
\usetikzlibrary{decorations.pathreplacing}
%\usetikzlibrary{backgrounds}
%
\begin{tikzpicture}[->,>=stealth',x=1.5cm, y=1.5cm, scale=1.1, transform shape]%,background rectangle/.style={fill=white}, show background rectangle]
\tikzset{dot/.style={circle,fill=#1,inner sep=0,minimum size=3pt}}

% centroids - that one that is moving
\coordinate (x) at (0,0);
\coordinate (cx) at (-1,1);
\coordinate (cj) at (3,1);

\begin{scope}[]
\clip (-1,2) rectangle (6,-1);
\draw[fill=black!10!white] (x) circle (5.4142);
\draw[fill=white] (x) circle (2.5858);
\end{scope}

\node[dot=black] at (x) {};
\node at (x) [above] {$\vec{x}$};
\node[dot=black] at (cx) {};
\node at (cx) [above] {$\vec{c}(\vec{x})$};
\node[dot=black] at (cj) {};
\node at (cj) [above] {$\vec{c}_j$};

\begin{scope}[rotate around={-10:(x)}]
\draw[dashed,<->] (x) -- +(2.5858,0);
\node at ($(x)+(1.2929,0)$) [below] {$\|\vec{c}(\vec{x})-\vec{c}_j\|-\|\vec{x}-\vec{c}(\vec{x})\|$};
\end{scope}

\begin{scope}[rotate around={10:(x)}]
\draw[dashed,<->] (x) -- +(5.4142,0);
\node at ($(x)+(2.7071,0)$) [below] {$\|\vec{c}(\vec{x})-\vec{c}_j\|+\|\vec{x}-\vec{c}(\vec{x})\|$};
\end{scope}

\end{tikzpicture}
  \usetikzlibrary{calc}
\usetikzlibrary{arrows}
\usetikzlibrary{decorations.pathreplacing}
%\usetikzlibrary{backgrounds}
%
\newcommand{\samples}{100}
%
\providecommand{\x}{\vec{x}}
\providecommand{\cj}{\vec{c}_j}
\newcommand{\dstxxx}{4.1231}
%
\begin{tikzpicture}[->,>=stealth',x=1.5cm, y=1.5cm, scale=1.1, transform shape]%,background rectangle/.style={fill=white}, show background rectangle]
\tikzset{dot/.style={circle,fill=#1,inner sep=0,minimum size=3pt}}

% centroids - that one that is moving
\coordinate (x) at (0,0);
\coordinate (cj) at (4,1);
\coordinate (move) at (0,-2);

\begin{scope}[]
\clip (0,1.5) rectangle (6.5,-1.5);
\draw[fill=black!10!white] (x) circle (\dstxxx+2);
\draw[fill=white] (x) circle (\dstxxx-2);
\end{scope}

\node[dot=black] at (x) {};
\node at (x) [above] {$\vec{x}$};
\node[dot=black] at (cj) {};
\node at (cj) [above] {$\vec{c}_j$};

\node[dot=black,rotate around={-20:(cj)}] at ($(cj)+(move)$) {};
\node[,rotate around={-20:(cj)},rotate=20] at ($(cj)+(move)$) [below] {$\cj'$};

\begin{scope}[rotate around={-10:(x)}]
\draw[dashed,<->] (x) -- +(\dstxxx-2,0);
\node at ($(x)+(\dstxxx/2-1,0)$) [below] {$\|\x-\cj\|-\|\cj-\cj'\|$};
\end{scope}

\begin{scope}[rotate around={10:(x)}]
\draw[dashed,<->] (x) -- +(\dstxxx+2,0);
\node at ($(x)+(\dstxxx/2+1,0)$) [below] {$\|\x-\cj\|+\|\cj-\cj'\|$};
\end{scope}

\end{tikzpicture}
 \end{center}
}

\block{Elkan's Algorithm \cite{elkan}}{
  \begin{itemize}
    \item One \emph{upper bound} $\ux$ -- distance to the assigned centroid.
    \item $k$ \emph{lower bounds} $\lxcj$ -- distance to each centroid.
  \end{itemize}
  \begin{center}
    \usetikzlibrary{calc}
\usetikzlibrary{arrows}
\usetikzlibrary{decorations.pathreplacing}
%\usetikzlibrary{backgrounds}
%
\begin{tikzpicture}[->,>=stealth',x=1.5cm, y=1.5cm, scale=2.0]%, transform shape]%,background rectangle/.style={fill=white}, show background rectangle]
\tikzset{dot/.style={circle,fill=#1,inner sep=0,minimum size=3pt}}

% centroids - that one that is moving
\coordinate (x) at (0,0);
\coordinate (c1) at (-0.5,0.5);
\coordinate (c2) at (3,1);
\coordinate (c3) at (5.3,-1);

\coordinate (cx) at (-0.5,0.5);
\coordinate (cj) at (3.3,1);

\begin{scope}[]
\clip (-1.2,1.5) rectangle (6,-1.2);

\draw[fill=black!20!white] (x) circle (10);
\draw[fill=black!10!white] (x) circle (5);
\draw[fill=white] (x) circle (3);

\draw[fill=black!10!white] (x) circle (1);

%\draw[fill=black!10!white] (x) circle (5.4142);
%\draw[fill=white] (x) circle (2.5858);
\end{scope}

\node[dot=black] at (x) {};
\node at (x) [above] {$\vec{x}$};
\node[dot=black] at (cx) {};
\node at (c1) [above right] {$\vec{c}_1$};
\node[dot=black] at (c2) {};
\node at (c2) [above] {$\vec{c}_2$};
\node[dot=black] at (c3) {};
\node at (c3) [above] {$\vec{c}_3$};

\draw[dashed,<->] (x) -- +(-1,0);
\node at (-0.5,0) [below] {$u(\vec{x})$};

\begin{scope}[rotate around={-10:(x)}]
\draw[dashed,<->] (x) -- +(3,0);
\node at ($(x)+(1.5,0)$) [below] {$l(\vec{x},\vec{c}_2)$};
\end{scope}

\begin{scope}[rotate around={10:(x)}]
\draw[dashed,<->] (x) -- +(5,0);
\node at ($(x)+(2.51,0)$) [below] {$l(\vec{x},\vec{c}_3)$};
\end{scope}

\end{tikzpicture}
  \end{center}
}

\block{Hamerly's Algorithm \cite{hamerly}}{
  2 point-center distance bounds per point:
  \begin{itemize}
    \item One \emph{upper bound} $u(\vec{x})$.
    \item One \emph{lower bound} $l(\vec{x})$ -- distance
      to the second-closest centroid.
  \end{itemize}
  \begin{center}
    \usetikzlibrary{calc}
\usetikzlibrary{arrows}
\usetikzlibrary{decorations.pathreplacing}
%\usetikzlibrary{backgrounds}
%
\begin{tikzpicture}[->,>=stealth',x=1.5cm, y=1.5cm, scale=2.0]%, transform shape]%,background rectangle/.style={fill=white}, show background rectangle]
\tikzset{dot/.style={circle,fill=#1,inner sep=0,minimum size=3pt}}

% centroids - that one that is moving
\coordinate (x) at (0,0);
\coordinate (c1) at (-0.5,0.5);
\coordinate (c2) at (3,1);
\coordinate (c3) at (5.3,-1);

\coordinate (cx) at (-0.5,0.5);
\coordinate (cj) at (3.3,1);

\begin{scope}[]
\clip (-1.2,1.5) rectangle (6,-1.2);

%\draw[fill=black!20!white] (x) circle (10);
\draw[fill=black!10!white] (x) circle (10);
\draw[fill=white] (x) circle (3);

\draw[fill=black!10!white] (x) circle (1);

%\draw[fill=black!10!white] (x) circle (5.4142);
%\draw[fill=white] (x) circle (2.5858);
\end{scope}

\node[dot=black] at (x) {};
\node at (x) [above] {$\vec{x}$};
\node[dot=black] at (cx) {};
\node at (c1) [above right] {$\vec{c}_1$};
\node[dot=black] at (c2) {};
\node at (c2) [above] {$\vec{c}_2$};
\node[dot=black] at (c3) {};
\node at (c3) [above] {$\vec{c}_3$};

\draw[dashed,<->] (x) -- +(-1,0);
\node at (-0.5,0) [below] {$u(\vec{x})$};

\begin{scope}[rotate around={-10:(x)}]
\draw[dashed,<->] (x) -- +(3,0);
\node at ($(x)+(1.5,0)$) [below] {$l(\vec{x})$};
\end{scope}

\end{tikzpicture}
  \end{center}
}

\column{0.25}

\block{Idea 1: Tighter Bound Updates}{
    The triangle inequality \cite{elkan, hamerly} is a worst-case update for
    the bounds. Points have locality, so when $\cj$ moves {\em away} from $\x$, the
    lower bound $\lxcj$ need not shrink. If the lower bound decreases
    slower, the bound is more useful.
  
    We use the upper bound to localize the cluster. 
    %Any point is at most $\ux$ from its closest centroid.
    Each point is at most
      \begin{equation*}
         \mci = \max_{\vec{x} \mid \vec{c}(\vec{x}) = \ci} u(\vec{x}).
      \end{equation*}
      from its closest centroid $\ci$.
  
     The update $\deltaxcj$ of $\lxcj$ must fulfill
       \begin{equation*}
         \deltaxcj \geq \distxcj - \distxcjp.
       \end{equation*}
     I.e. the update is at least the difference between
     the old and the new distance. Now focus on function
      $\distxcj - \distxcjp = f(\x)$.
}

\block{Lower Bound Update Calculation}{
If we fix $f(x) = \distxcj - \distxcjp = z$, we obtain a hyperbola. For any point on or above the hyperbola we can use $\deltaxcj = z$ as the update of the lower bound $\lxcj$.

The optimal value of $z$ could be found if the hyperbola touches the sphere that contains the cluster. This would lead to too costly calculations. Therefore instead we let asymptote of the hyperbola touch the sphere and use a bit bigger value of $z$.

\begin{lemma}
Suppose that $\x \in \R^2$, $\ux \leq r \in \Rpz$ and $\ci = (c_{ix}, c_{iy})$,
where $c_{ix} > r$ and $c_{iy} \leq r$. Let $\cj=(0,1)$ and $\cj'=(0,-1)$. Then
\begin{equation*}
    \deltaxcj =
        2\frac{
            c_{ix} r
            -
            c_{iy} \sqrt{\| \ci \|^2 - r^2}
        }{
             \| \ci \|^2
        }
\end{equation*}
is a valid update of the lower bound $\lxcj$.
\end{lemma}
  \begin{center}
    \usetikzlibrary{calc}
\usetikzlibrary{arrows}
\usetikzlibrary{decorations.pathreplacing}

\newcommand{\samples}{100}

\providecommand{\x}{\vec{x}}
\providecommand{\cj}{\vec{c}_j}
\providecommand{\ci}{\vec{c}_i}


\begin{tikzpicture}[x=1in, y=1in, scale=2.5, line width=1pt]

% centroids - that one that is moving
\coordinate (c) at (0,1);
\coordinate (c') at (0,-1);
\tikzset{dot/.style={circle,fill=#1,inner sep=0,minimum size=5pt}}
\node[dot=black] at (c) {};
\node at (c) [above right] {$\cj$};
\node[dot=black] at (c') {};
\node at (c') [below right] {$\cj'$};

% draw the hyperbola
\begin{scope}[]
\clip (-1,-1.25) rectangle (1,1.25);
\draw[] plot[variable=\t,samples=\samples,domain=-64:64] ({0.7071*tan(\t)},{-0.7071*sec(\t)});
\end{scope}

% draw axis
\draw (-1,0) -- (2,0);
\draw (0,-1.2) -- (0,1.2);

% draw the asymptote
\draw[dashed] (0,0) -- (1.2,-1.2);
% draw the s vector
\coordinate (s) at (0.9,-0.9);
\draw[thick,->,line width=2pt] (0,0) -> (s);
\node[dot=black] at (s) {};
\node at (s) [above=5pt] {$\vec{s}$};

% the center ci
\coordinate (ci) at (0.5,0.5);
\node[dot=black] at ($(s)+(ci)$) {};
\node at ($(s)+(ci)$) [below right] {$\ci$};
% the circle
\draw ($(s)+(ci)$) circle (0.7071);

% the radius
\draw[color=gray, dashed] ($(s)+(ci)$) -- (s);
\node[color=gray] at ($(s)+0.5*(ci)$) [above left] {$r$};

% axis of hyperbola
%\draw[color=gray,decorate, decoration={brace, amplitude=4pt}] (0,-0.7071) -- (0,0);
%\node at ($0.35*(c')$) [left = 4pt] {$a$};
%\draw[color=gray, dashed] ($0.5*(c')$) -- +(0.5,0);
%\node[color=gray] at ($0.5*(c')+(0.25,0)$) [above] {$b$};

% point x
\coordinate (x) at (1,-0.1);
\node[dot=gray] at (x) {};
\node[color=gray] at (x) [below] {$\x$};

\end{tikzpicture}
  \end{center}
}


\column{0.25}
\block{Idea 2: Avoid $\Theta(k)$ Work in the Innermost Loop}{
    Hamerly's algorithm finds the distance to
    the second-closest centroid in its innermost loop.
    If we know which are the two closest centroids, no other 
    distances need to be calculated.
  \begin{definition}
    Centroid $\cj$ is a \emph{neighbor} of $\ci$ if it is one of the two closest
    for any point assigned to~$\ci$.
  \end{definition}
  \begin{theorem}
    Any neighbor $\cj$ of a centroid $\ci$ must fulfill
    \begin{equation}
      \mci + \sci \geq \frac{1}{2} \| \ci - \cj \|
      \label{eq:neighborcond}
    \end{equation}
    where $\sci$ is the distance from $\ci$ to its closest centroid.
  \end{theorem}
  
    Suppose \eqref{eq:neighborcond} is violated (for a non-neighbor).
    Then for any $\x$ assigned to $\ci$:
    \begin{itemize}
      \item $\ci$ is closer to $\x$ than $\cj$ and
      \item the closest other centroid to $\ci$ is closer to $\x$ than $\cj$.
    \end{itemize}
    Therefore we do not need to consider non-neighbor $\cj$ in the innermost
    loop for \emph{all} points in the cluster assigned to $\ci$.
}

\block{Experimental results}{
 Sources are available at \url{https://github.com/petrrysavy/baylorml} [branches \texttt{modified\_update} and \texttt{multithreaded}]. Implementation is based on code by G. Hamerly and J. Drake. We tested the algorithms on synthetic and real-world datasets.
 
 Changes require operations per iteration and pair of centroids. There is no additional work per point. Therefore there is low risk of slowing down the algorithm while the runtime may improve several times. Changes work better in lower dimension and when data contain a natural clustering. Algorithms become up to 8 times faster that the original versions and 300 times faster than Lloyd's algorithm.
}

\block{References}{
    \footnotesize
	\nocite{chapter, hamerly, elkan, lloyd}
    \bibliographystyle{plain}
    \bibliography{files/thesis}
}

\end{columns}

\end{document}
